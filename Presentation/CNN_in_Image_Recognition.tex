\documentclass[12pt]{beamer}
\usepackage{amsmath}
\usepackage{amssymb}
\usepackage{times}
\usepackage{graphicx}

\usetheme{Warsaw}
\setbeamertemplate{footline}[frame number]

% ---------- TITLE & SUBTITLE (SMALLEST SIZE) ----------
\title{\small The Brain of AI: Exploring the World of Neural Networks}
\subtitle{\footnotesize How artificial neurons power today’s intelligent technologies.}

\author{\small Misbah Tasleem NS \\ 
	\scriptsize 5th Sem CSE(AI \& ML)}

\institute{
	\scriptsize Under the guidance of\\[0.4em]
	\small Dr. Poorna Chandra {\tiny (Mentor)} \hspace{1cm} 
	Dr. Sachin Kumar {\tiny (HOD)}\\[0.4em]
	\small Dept. of CSE(AI \& ML)\\
	\scriptsize Jain College of Engineering, Belagavi
}


\date{\footnotesize\today}

% ---------- LOGOS (REDUCED SIZE + WILL FIT) ----------
\titlegraphic{
	\vspace{0.4em}
	\includegraphics[height=1.2cm]{vtu-logo.jpg} \hspace{0.5cm}
	\includegraphics[height=1cm]{jce.jpg}
}


% ---------- LOGOS (REDUCED SIZE + WILL FIT) ----------
\titlegraphic{
	\vspace{0.5em}
	\includegraphics[height=1.5cm]{vtu-logo.jpg} \hspace{0.7cm}
	\includegraphics[height=1.2cm]{jce.jpg}
}






\begin{document}
	
	%-------------------Title Slide-------------------
	\begin{frame}
		\maketitle
	\end{frame}
	
	%-------------------Frame 1: Points 1 and 2-------------------
	\begin{frame}{Evolution of Neural Networks (Part 1)}
		\begin{block}{1. What is a Neural Network?}
			A system that takes inputs (images, text) and produces useful outputs (e.g., recognition, translation).
		\end{block}
		
		\begin{block}{2. First Wave (1940s–1960s): Early Neural Models}
			\begin{itemize}
				\item McCulloch \& Pitts (1943) created the first artificial neuron.
				\item Rosenblatt (1957) introduced the perceptron, the first learnable model.
				\item Perceptrons couldn’t solve complex problems, leading to the first AI winter (Minsky \& Papert, 1969).
			\end{itemize}
		\end{block}
	\end{frame}
	
	%-------------------Frame 2: Points 3 and 4-------------------
	\begin{frame}[allowframebreaks]{Evolution of Neural Networks (Part 2)}
		\begin{block}{3. Second Wave (1980s–1990s): Backpropagation \& CNNs}
			\begin{itemize}
				\item Backpropagation (1986) enabled learning in multilayer networks.
				\item LeNet (1989) by LeCun et al. recognized handwritten ZIP codes.
				\item Neural networks lost popularity due to limited computing power and stronger traditional ML methods.
			\end{itemize}
		\end{block}
		
		\begin{block}{4. Third Wave (2000s–Present): Deep Learning Era}
			\begin{itemize}
				\item Improved algorithms, large datasets, and GPUs fueled renewed interest.
				\item AlexNet (2012) popularized deep learning by winning ImageNet.
				\item Neural networks now dominate across  many applications and industries.
			\end{itemize}
		\end{block}
	\end{frame}
	
	
	
	\begin{frame}{Biological Neuron}
		\begin{itemize}
			\item A biological neuron has:
			\begin{itemize}
				\item Dendrites – receive input signals.
				\item Cell body – processes inputs.
				\item Axon – sends outputs.
				\item Synapses – connect neurons.
			\end{itemize}
		\end{itemize}
		
		\begin{figure}
			\centering
			\includegraphics[width=0.6\textwidth]{neuron.png}
			\caption{Structure of a Biological Neuron}
		\end{figure}
	\end{frame}
	
	\begin{frame}{Artificial Neuron (Perceptron)}
		\begin{itemize}
			\item The perceptron is inspired by how a biological neuron works.
			\item It performs:
			\begin{itemize}
				\item Input signals ($x_1, x_2, \ldots$) — features.
				\item Weights ($w_1, w_2, \ldots$) — importance of inputs.
				\item Summation — adds weighted inputs.
				\item Bias ($b$) — adjusts sensitivity.
				\item Activation function — decides output.
			\end{itemize}
		\end{itemize}
		
		\begin{figure}
			\centering
			\includegraphics[width=0.6\textwidth]{perceptron.png}
			\caption{Structure of a Perceptron}
		\end{figure}
	\end{frame}
	\begin{frame}{How It Mimics the Biological Neuron}
		
		\begin{itemize}
			\item Dendrites: Inputs ($x_1, x_2, \ldots$)
			\item Synaptic Strength: Weights ($w_1, w_2, \ldots$)
			\item Cell Body (Summing signals): Summation Function ($\sum x_i w_i$)
			\item Axon Hillock (Decision point): Activation Function (Sign/Step function)
			\item Axon (Signal path): Output ($y$)
			\item Threshold Potential: Bias ($b$)
		\end{itemize}
		
	\end{frame}
	
	\begin{frame}{Example: Perceptron with 2 inputs}
		For weights $w_0 = 0.9$, $w_1 = -0.6$, and $w_2 = -0.5$:
		
		\medskip
		
		When tested with all combinations of inputs $(-1, 1)$, the perceptron behaves like a NAND gate.  
		This means perceptrons can simulate basic Boolean logic.
		
		\begin{figure}
			\centering
			\includegraphics[width=0.6\textwidth]{2perceptron.png}
			\caption{Structure of a 2-input Perceptron}
		\end{figure}
	\end{frame}
	\begin{frame}{Perceptron Learning Algorithm}
		The perceptron can learn correct weights automatically through supervised learning:
		\begin{itemize}
			\item Initialize weights randomly.
			\item Pick an input–output pair.
			\item Compute the perceptron’s output.
			\item If it’s wrong, update each weight using:
			\[
			w_i = w_i + (\text{learning\_rate}) \times y \times x_i
			\]
			\item Repeat until all outputs are correct.
		\end{itemize}
		This process adjusts the weights gradually, allowing the perceptron to learn patterns.
	\end{frame}
	\begin{frame}{Limitations}
		The perceptron can only classify linearly separable data (data that can be divided by a straight line).
		For example:
		•	Works for NAND or AND functions
		
		•	Fails for XOR, since XOR cannot be separated by a single straight line.
		
		\textbf{Overcoming the Limitation:}
		\begin{itemize}
			\item •	Combine multiple perceptrons in layers.
			\item •	The first layer (hidden layer) separates parts of data.
			\item •	The second layer (output neuron) combines them.
		\end{itemize}
		
		This forms a multi-layer neural network (basis of modern deep learning).
	\end{frame}
	\begin{frame}[allowframebreaks]{Journey: From Perceptron to Modern CNNs}
		\begin{itemize}
			\item \textbf{1. Single Neuron (Perceptron) — 1957}
			\begin{itemize}
				\item Takes inputs, multiplies by weights, adds bias, passes through activation.
				\item Can only solve linearly separable problems (simple yes/no).
			\end{itemize}
			
			\item \textbf{2. Multi-Neuron (MLP) — 1980s}
			\begin{itemize}
				\item Multiple neurons connected in layers; uses backpropagation to learn.
				\item Not efficient for images; huge number of weights if image is large.
			\end{itemize}
			
			\item \textbf{3. Local Connectivity — Late 1970s–1980s}
			\begin{itemize}
				\item Each neuron connects only to a small patch of the input (like human vision).
				\item Reduces computation and focuses on local patterns.
			\end{itemize}
			
			\item \textbf{4. Neocognitron — 1980 (Fukushima)}
			\begin{itemize}
				\item Introduced convolution-like layers to detect patterns.
				\item Added pooling-like layers to reduce size and retain important info.
			\end{itemize}
			
			\item \textbf{5. LeNet — 1989–1998}
			\begin{itemize}
				\item First practical CNN combining convolution, pooling, and fully connected layers.
				\item Used for handwritten digit recognition (ZIP codes, cheques).
			\end{itemize}
			
			\item \textbf{6. Modern CNNs — 2012 Onwards}
			\begin{itemize}
				\item Deep networks with GPUs and large datasets (AlexNet, VGG, ResNet).
				\item Can learn complex features efficiently and power today’s computer vision.
			\end{itemize}
		\end{itemize}
	\end{frame}
	\begin{frame}{CNN in Image Recognition 
			\newline1. Intuition \& Overview}
			\begin{itemize}
		\item A CNN (Convolutional Neural Network) processes images by learning features (like edges, shapes, textures) in successive layers.
		
		\item Early layers detect simple patterns (edges, corners); deeper layers combine those into more complex features (ears, face, tail) and finally the network decides what the object is.
		
		\item The advantage: CNNs learn where and what in the image is important automatically, using filters/kernels, pooling, and fully connected layers.
			\end{itemize}
		\end{frame}
		
		
		
		%-------------------2. Example Setup-------------------
		\begin{frame}{2. Example Setup}
			Let’s say our dataset is small and we want to classify images into two classes: cat or dog. \\
			Each image is 64 × 64 pixels, with 3 color channels (RGB). So the input shape is 64 × 64 × 3.
			\begin{figure}
				\centering
				\includegraphics[width=0.7\textwidth]{cnn1.png}
				\caption{Overview of a Convolutional Neural Network}
			\end{figure}
			
		\end{frame}
		
		%-------------------3. Layer-by-Layer Walkthrough-------------------
		\begin{frame}{3. Layer-by-Layer Walkthrough}
			Here’s a simple CNN architecture we use as our example:
			
			\begin{itemize}
				\item Input layer
				\item Convolutional layer + ReLU
				\item Max Pooling
				\item (Possibly more conv + pooling layers)
				\item Flatten layer
				\item Fully connected (Dense) layer(s)
				\item Output layer (softmax or sigmoid)
			\end{itemize}
			
			\begin{figure}
				\centering
				\includegraphics[width=0.8\textwidth]{cnn2.png}
				\caption{Simple CNN Architecture Example}
			\end{figure}
		\end{frame}
		
		%-------------------A. Input Layer-------------------
		\begin{frame}{A. Input Layer}
			Receives the raw pixels: 64 × 64 × 3
			
			No weights here — it’s just the data.
		\end{frame}
		
		%-------------------B. Convolutional Layer-------------------
		\begin{frame}{B. Convolutional Layer}
			\begin{itemize}
			\item The convolution layer uses small filters (kernels) like 3×3 or 5×5.
			
			\item These filters slide over the image from left to right, top to bottom.
			
			\item At every position, the filter and the image region are multiplied and added → gives one value.
			
			\item Doing this for the whole image forms a feature map.
			
			\item These feature maps highlight important patterns such as:
			
			vertical edges,horizontal edges,corner,textures
			
		   \item Convolution helps CNN understand the image locally (small region at a time).
		\end{itemize}
		\end{frame}
		
		%-------------------C. Activation Function-------------------
		\begin{frame}{C. Activation Function (ReLU)}
			\begin{itemize}
				
		 \item After convolution, you apply ReLU (Rectified Linear Unit):
		\item 	It keeps positive values as they are and makes all negative values = 0.
			
		\[ 
			\text{ReLU}(x) = \max(0, x)
			\]
			
			\item This introduces non-linearity, making the model more expressive (can capture more complex patterns).
		\end{itemize}
		\end{frame}
		
		%-------------------D. Pooling Layer-------------------
		\begin{frame}{D. Pooling Layer (Max Pooling)}
		\begin{itemize}
			\item Pooling reduces the size of the feature maps.
			
			\item Max Pooling takes the maximum value from each small region (e.g., 2×2 window).
			
			\item Helps keep the important features while removing unnecessary details.
			
			\item Makes the model faster and reduces computation.
			
		    \item Provides spatial invariance (small changes in image don’t affect output).
		\end{itemize}
		\end{frame}
		
		%-------------------E. Additional Conv + Pooling Layers-------------------
		\begin{frame}{E. Additional Conv + Pooling Layers (optional)}
			\begin{itemize}
				\item Adding more conv and pool layers makes the CNN deeper, so early layers learn simple things (edges), and later layers learn complex things (shapes and objects).
				
				\item More layers can improve accuracy, but they also take more time to train and more computation power .
			\end{itemize}
		\end{frame}
		
		%-------------------F. Flatten Layer-------------------
		\begin{frame}{F. Flatten Layer}
		\begin{itemize}
			\item After convolution and pooling, the output is still in 2D or 3D form (like 7×7×64).
			Flatten layer converts this into a single long 1-dimensional vector.
			
			\item  This step is needed because the next part of the model (fully connected layers) can only take 1D input.
			
			\item Example:\newline
			If the feature map size is 7 × 7 × 64, flattening it gives 3136 values in one row.
		\end{itemize}
		\end{frame}
		
		%-------------------G. Fully Connected (Dense) Layer-------------------
		\begin{frame}{G. Fully Connected (Dense) Layer(s)}
			This is like a classic neural network layer.
			
			Suppose we have one dense layer of 128 neurons. Each neuron connects to all flattened features.
			
			Activation (e.g. ReLU) is applied again.
			
			\textbf{What it does:} Combines all features to find higher-level decision boundaries.
		\end{frame}
		
		%-------------------H. Output Layer-------------------
		\begin{frame}{H. Output Layer}
			\begin{itemize}
				\item The output layer gives the final prediction of the CNN.
				We use Softmax or sigmoid function  to convert the outputs into probabilities for each class.
				\item If we use Softmax
				
				We need 2 neurons (one for Cat, one for Dog).
				
				Softmax gives two probabilities that add up to 1.
				\item Since we classify Cat vs Dog, the model gives two probabilities, and the higher one tells whether the image is a Cat or a Dog.
				
				
				\item  If we use Sigmoid
				
				We need 1 neuron.
				
				Sigmoid gives one probability (example: greater than  0.5 = Dog, less than 0.5 = Cat).
			\end{itemize}
		\end{frame}
		
		%-------------------4. Forward + Backprop-------------------
		\begin{frame}{4. How It Works Together (Forward + Backprop)}
			\textbf{Forward Pass:}
			
			Image → conv → ReLU → pooling → (repeat) → flatten → dense → output (probability).
			
			\textbf{Loss Computation:}
			
			Compare output with true label (cat or dog), compute loss (e.g. cross-entropy).
			
			\textbf{Backpropagation:}
			
			Compute gradients of loss w.r.t. all weights (in conv filters, dense layers).
			
			Update weights (filters and dense weights) using gradient descent (or optimizer like Adam).
			
			This trains the network to learn filters that detect features suited to distinguishing cats vs dogs.
		\end{frame}
		
		
		
		
		%-------------------6. What Each Layer Learns-------------------
		\begin{frame}{5 . What Each Layer Learns}
			First conv layer: edges, simple textures
			
			Second conv layer: combinations of edges → curves, small shapes
			
			Dense layers: combine these shapes to detect parts like ears, face, fur
			
			Output: final decision: cat or dog
		\end{frame}
		
		%------------------- Slide: What is LeNet -------------------
		\begin{frame}{What is LeNet?}
			\begin{itemize}
				\item LeNet is a neural network (a type of AI model) created by Yann LeCun.
				\item Its job: look at a picture of a digit and tell which number it is.
				\item It was used for tasks like reading:
				\begin{itemize}
					\item ZIP codes on letters
					\item Numbers written on bank cheques
				\end{itemize}
			\end{itemize}
		\end{frame}
		
		%------------------- Slide: How LeNet Sees an Image -------------------
		\begin{frame}{How LeNet Sees an Image}
			Think of recognizing a number the way you scan a page:
			\begin{itemize}
				\item First you look at tiny details (lines, small curves)
				\item Then you join these small parts to see bigger shapes
				\item Finally, your brain says: “This looks like a 3 or a 7.”
			\end{itemize}
			\vspace{0.5em}
			LeNet works in the same way.
		\end{frame}
		
		%------------------- Slide: How LeNet is Built -------------------
		\begin{frame}{How LeNet is Built (Simple Analogy)}
			LeNet has layers, like steps in a process. Each layer does something special.
			
			\begin{itemize}
				\item Step 1: Looking at small parts (Convolution Layer)
				\item Step 2: Keeping important things (Pooling Layer)
				\item Step 3: Understanding bigger shapes (More Convolution + Pooling)
				\item Step 4: Final decision-making (Fully Connected Layers)
			\end{itemize}
		\end{frame}
		
		%------------------- Slide: Step 1 -------------------
		\begin{frame}{Step 1: Convolution Layer}
			This layer looks at tiny pieces of the image:
			\begin{itemize}
				\item little lines
				\item edges
				\item corners
			\end{itemize}
			It works like zooming in to find small clues.
		\end{frame}
		
		%------------------- Slide: Step 2 -------------------
		\begin{frame}{Step 2: Pooling Layer}
			From all the details, it keeps only the strongest or most important ones.
			\begin{itemize}
				\item Similar to taking notes – you only keep key points.
				\item Pooling reduces the size but keeps meaningful information.
			\end{itemize}
		\end{frame}
		
		%------------------- Slide: Step 3 -------------------
		\begin{frame}{Step 3: Understanding the Bigger Picture}
			The model now combines small clues to recognize bigger shapes:
			\begin{itemize}
				\item Curves of “3”
				\item Straight lines of “1”
				\item Round shapes of “0”
			\end{itemize}
			More convolution + pooling improves pattern recognition.
		\end{frame}
		
		%------------------- Slide: Step 4 -------------------
		\begin{frame}{Step 4: Fully Connected Layers}
			This step works like the brain thinking:
			
			\begin{block}{}
				“Based on all the features, this looks like a 7.”
			\end{block}
			
			The final layer gives the answer:
			\begin{itemize}
				\item Which digit (0–9) the image represents
			\end{itemize}
		\end{frame}
		
		%------------------- Slide: Why LeNet Was Special -------------------
		\begin{frame}{Why LeNet Was Special}
			\begin{itemize}
				\item Before LeNet, computers struggled to read handwritten text.
				\item LeNet learned from examples, just like humans do.
				\item It didn’t need someone to manually design rules.
				\item LeNet learned its own rules from data.
			\end{itemize}
			
			This idea became the foundation for modern AI systems like:
			\begin{itemize}
				\item Self-driving cars
				\item Face recognition
				\item Advanced image processing
			\end{itemize}
		\end{frame}
		
		%------------------- Slide: Full Process -------------------
		\begin{frame}{LeNet in One Simple Line}
			\centering
			\textbf{
				Image → Find small patterns → Pick important patterns → Combine them → Decide the number
			}
		\end{frame}
		
		
		\begin{frame}[allowframebreaks]{References}
			\footnotesize
			\begin{thebibliography}{99}
				
				\bibitem{lenet}
				Yann LeCun.  
				\textit{LeNet Convolutional Neural Network}.  
				Available at: \url{http://yann.lecun.com/exdb/lenet/index.html}
				
				\bibitem{nvidia}
				Magnus Ekman.  
				\textit{Learning Deep Learning: Theory and Practice of Neural Networks, Computer Vision, Natural Language Processing, and Transformers Using TensorFlow}.  
				NVIDIA Deep Learning Institute, 2021.
				
			\end{thebibliography}
		\end{frame}
		
		
		
	\end{document}